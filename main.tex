\documentclass[10pt, twocolumn]{article}
\usepackage[total={6in, 8in}, margin=0.75in]{geometry}
\usepackage{booktabs} % for nice tables
\usepackage{longtable}
\usepackage{lmodern}
\usepackage{listings}
\usepackage{hyperref}
\usepackage{graphicx}
\usepackage{subcaption}
\usepackage{wrapfig}
\usepackage{cite}
\usepackage{url}
\usepackage{microtype}
\usepackage[toc,page]{appendix}
%\usepackage{lineno}

\usepackage{amssymb}
\usepackage{tabularx}
%\usepackage[paper=letterpaper,top=1in,bottom=1in,left=1in,right=1in]{geometry}

\usepackage[T1]{fontenc}
\usepackage[sc]{mathpazo}
\linespread{1.02}

\setlength\parindent{0pt}
\setlength{\parskip}{7pt plus4mm minus3mm}
\newcommand{\block}{\footnotesize $\blacksquare$ \normalsize}
%\pagestyle{empty}
\newcommand{\squeezeup}{\vspace{-5.5mm}}
%\linenumbers

\bibliographystyle{plain}

\begin{document}
\title{Known Vs Novelty: Exploring two Programming Abstractions for Experimental HEP Analyses}
\author{Saba Sehrish \\ Computer Science Researcher\\ Fermi National Accelerator Laboratory \\ ssehrish@fnal.gov}
\date{}
\maketitle

\thispagestyle{empty}

\section*{Abstract}
\squeezeup
Analyzing petabytes of experimental physics data requires repeated expensive evaluation of selection and filtering criteria. We describe how MPI and Spark can improve time-to-physics for interactive analysis on HPC resources. We evaluate the performance benefits of holding full datasets in memory and the productivity benefits to physicists when using high-level abstractions. 
\\
Track: Data Science  \\
Audience: Beginner, Intermediate technical talk \\
\squeezeup
\section{Introduction}
\label{sec:intro}
\squeezeup

%Experimental High Energy Physics (HEP) deals with the understanding of fundamental particles and the interactions between them.
%Experimental HEP is a compute- and data-intensive statistical science; a large number of interactions 
%must be analyzed to discover new particles or to measure the properties of known particles. 
%For example, data from over $300$ trillion ($3\times10^{14}$) proton-proton collisions at the Large Hadron Collider (LHC) were analyzed for the Higgs boson discovery.
 %approximately $1$ in $10^{13}$ LHC collisions yielded a distinguishable Higgs boson~\cite{higgsboson1}. 
%The size of the data sample (25 petabytes per year) required the use of a worldwide computing grid, 
%comprised of 170 computing facilities across 42 countries~\cite{lhcgrid}. 
%Future HEP experiments will bring in even more data, and processing and 
%analyzing will be more challenging. For example, while the LHC generates up to a billion collisions per second; the High Luminosity-LHC~\cite{hllhc} will generate 5 times this rate. These larger data samples will be needed to obtain a 
%deeper understanding of the Higgs boson and its implications for the fundamental laws of nature. 
 
Experimental High Energy Physics (HEP) is a compute- and data-intensive statistical science; 
a large number of interactions ($\approx$petabytes of data)
must be analyzed to discover new particles or to measure the properties of known particles. 
For example, data from over $300$ trillion ($3\times10^{14}$) proton-proton collisions were analyzed for the Higgs boson discovery at the 
Large Hadron Collider (LHC). 
The current LHC generates up to one billion collisions per second. When the High Luminosity-LHC~\cite{hllhc} turns on, it will generate 5 times this rate.
With this increase in data volume, the current tools and resources will be insufficient to 
provide low-latency interactive analysis. 

A typical complete HEP Analysis is an iterative process where low-latency and interactivity are key elements in scientific discovery. 
Analyses are IO bound and consist of several intermediate steps, each requiring storage of 
smaller reduced data sets on disk, which can be later accessed interactively for final analysis. 
These intermediate steps consist of applying selection algorithms, calculations of statistical summaries, 
and exploratory plotting of the relevant summaries. 
These steps can take anywhere from days to weeks. 
Storing intermediate results into files and then into data handling systems 
add to the processing time. 

%CMS example
In the CMS Dark Matter search for new types of elementary particles,
the data volume to be analyzed is about 200 terabytes of the 2015 dataset. 
 %and contains both data and Monte Carlo simulations. 
 This dataset is expected to significantly grow in the future as the HL-LHC approaches. 
 The data is reduced to couple of gigabytes to facilitate final analysis.  
Similarly, in an intensity frontier experiment, NOvA, a neutrino analysis selects 33 events (in megabytes range) out of a billion events (in terabytes range) for 
parameter estimation and model comparison studies. 

%a particular analysis 
 %performs selection on a billion events (in terabytes range), and uses the resulting 33 events (in megabytes range) with fewer 
 %attributes to do 
%The workflow starts with partly-reconstructed detector data,
%and with simulated neutrino events.
%These inputs come in the form of many files,
%with an aggregate size in the petabyte range.
%The first (reconstruction) step involves
%an \art framework program,
%run in many independent jobs,
%and is used to perform tasks such as particle identification
%and (neutrino) event identification.
%The output of these program runs is a large number of large n-tuple files,
%containing reduced data
%for each identified candidate neutrino event.
%The aggregate data size of the n-tuple files
%is in the terabyte range.
%The second (filtering) step
%involves a serial \cpp program,
%reads the large n-tuples,
%and performs a strict event selection
%and further reduces the data describing an event to a few numbers.
%The output of this filtering step
%is in the range of a few megabytes.
%The final (fitting) step consists
%of running a serial implementation
%of the Feldman-Cousins algorithm
%at each of many points in the mixing parameter space,
%chosen on a fixed grid,
%%to map out contour plots describing the correlated parameter estimates.
%\begin{figure*}[htbp]
%\begin{center}
% \includegraphics[scale=0.5]{analysisworkflow}
%\caption{Different steps in an analysis workflow and data sizes after each step }
%\label{fig:flow}
%\end{center}
%\end{figure*}
To 
improve time-to-physics and enable interactive analysis without significant overhead of bookkeeping and intermediate 
files, 
we need to explore alternate approaches such as in-memory data processing. The new large-scale, state-of-the-art computing systems provide tremendous compute and memory resources causing us to rethink the traditional methods of doing HEP data analytics.  
For example, Cori, a supercomputer at NERSC, provides total memory of 203 terabytes for the Haswell nodes with sustained application performance of 83 TFlop/s. 
However,  choosing programming abstractions to efficiently access and analyze data on these big compute machines remains a challenge. 
The traditional approach of using MPI on a system like Cori burdens the programmer with fine grain details and control of the application to utilize the system, such as data and task assignment, choice of memory technology, etc. Spark, an industry de-facto standard, the new approach of using high level programming abstractions on distributed datasets, has been recently made available on Cori. It is, however, neither designed for scientific applications nor to exploit the high performance computing platforms that are 
available to the scientific community.

In this talk, we present a thorough comparison of performance and usability aspects of 
both the traditional and the industry standard approach by implementing use cases from HEP.  
More specifically, we will focus on the challenges related to implementing selection and 
filtering criteria on billions of events using Apache Spark~\cite{spark,spark1} and MPI by running tests at NERSC~\cite{nersc-spark}. 
%Our work is unique because we have been using both MPI and Spark on HPC resources for HEP analyses. . 
\squeezeup
%\squeezeup
\section{Features of selection and filtering and our approach}
%\squeezeup


Our input data consists of events and their attributes. 
The event is the smallest unit of data representing a complete readout of the full
detector (whether real or simulated). 
The structured event data is converted to a simplified tabular data structure for analysis tasks, called as n-tuples. 
These n-tuples have a simple flat structure of vectors of basic types like integers and floats. 
If an event is representing collision data, then 
each row of a table holds information about different particles (photons, electrons, taus) and 
their properties (momentum, etc) in an event. Often, the n-tuples are still too big for interactive 
analysis $\approx$ terabytes. Therefore, the contents and the number of events are 
reduced to couple of gigabytes. Eventually, quantities from the final n-tuple are aggregated 
and plotted as histograms. 
%The time scale of the complete Dark Matter workflow 
%can range from days to weeks, depending on the number of events needed for analysis.

A typical analysis task looks at all the event data, which includes data about 
all the particles that were formed as a result of collision and searches for 
interesting events based on signatures specified by scientists. 
The selection and filtering criteria involves reducing the event count, 
as well as the attributes. This directly translates to range queries 
across multiple variables. Such queries are not only hard to 
describe by just SQL declarative statements, but also 
challenging to deal with using the obvious relational database approach, 
unless we main index for each column involved in range query or do a 
full scan of data. Both options are inefficient in terms of space utilization 
as well performance. 

\textit{Example Query:} Find all the events that have:
1) missing transverse energy greater than 200 GeV and 
2) at least one electron candidate with: transverse momentum > 200 GeV, eta in the range of -2.5 to 2.5
and good electron quality: qual > 5
For each selected event record, select: missing transverse energy , the leading electron transverse momentum

To benefit from big compute and memory architecture, we represent our data 
in HDF5 format, which is highly optimized for access in these machines. 
In Spark, we use the combination of Scala functional programming with Spark SQL 
allows us to describe the queries to operate on dataset as a whole. 
We used the User-defined Functions (UDF) provided by Spark SQL to define new column based 
functions. We implemented several UDFs, and a few were using inputs from tens of 
other columns, see~\cite{spark-hdf5-cms} for details. The drawback of using UDFs is they are treated as black box code, 
and Spark doesn't try to optimize these~\cite{spark-notes}. In some cases, we divided required filter conditions 
into multiple equivalent conditions to simplify UDFs. 
Several filter operations per DataFrame are defined; but all these are lazy evaluated. 
In our filter implementation, we used several conditionals, 
\texttt{select} queries, \texttt{groupby}, and \texttt{aggregations (sum, count)}. 
After the aggregation operations, we are usually left with one particle per event in 
all the particles DataFrames. 
We used \texttt{SQL select} queries to join columns from different DataFrames.
The analysis was implemented in Apache Spark 2.0 DataFrame API using Scala.  

The MPI implementation used parallel read with in each file utilizing the Python interface 
for reading HDF5 files, \texttt{h5py}~\cite{h5py}, with MPI/IO support (\texttt{mpi4py}~\cite{mpi4py}). 
In MPI, we used \texttt{NumPy}~\cite{numpy} arrays to hold the data in memory and 
\texttt{MPI.COMM\_WORLD.Reduce()} with op as \texttt{MPI.SUM}. 
The interface provided by \texttt{h5py}, \texttt{mpi4py}, and \texttt{numpy} 
is simple to use, and offers a competitive approach to the high level API offered by Spark. 
%In MPI, we make use of the numpy arrays and define operations on the arrays, 
%and use MPI for communication. 

We implement this problem in two phases, one is read data, and second is process data. 

% talk about tuning 
% two step problem: optimize read, initial read form disk and effecient layout in memory
% stripe size equal to chunk size 
% encode multi-dimensional queries on the tabular data
% talk about how numpy provides competitive approach, 
% however once data is in memory Spark is comparable ... 




%\squeezeup
\section{Comparing two approaches}
%\squeezeup
 \paragraph{On-disk data} 
%Our first step is to convert the data to a form that 
%can be readily read in Spark DataFrames as shown in Figure~\ref{fig:flow}. 
%The CMS data is stored in the form of ROOT trees~\cite{root}, which is the most common 
%data format in HEP, which are not accessible in Spark. 
%The ROOT data format is hierarchical and complex and Spark DataFrames are tabular.  
%Hence, we need flattened tables to effectively use the API. 
%Additionally, we are working to use Avro and HDF5 data formats for this project. 
HDF5 is used as the input data format for both MPI and Spark implementation. 
It is one of the tuned and supported parallel I/O libraries available on HPC machines. 
Parallel HDF5 is tightly coupled with MPI, and several optimization are 
available to efficiently read different data access patterns. 
In Spark, reading HDF5 data is non-trivial, and it requires several steps 
to read it. We implemented a custom reader to achieve the required functionality. 

\paragraph{In-memory data}
We read data into Spark DataFrames and numpy arrays. 
pySpark uses numpy arrays in their implementation. 

Currently we read data in multiple logical tables, e.g. all information regarding 
a given reconstructed particle type can be found in one table. The nature of this analysis is to 
combine the data from different tables 
and make several histograms for different particles properties. 
In our experience, if we read data from multiple DataFrames, 
adding a new column to the DataFrame from another 
DataFrame is not supported within Spark. Spark only supports adding new columns within the same DataFrame. 
We also use inner joins to combine the DataFrames, but the overhead of the 
\texttt{join} operation is extremely overwhelming and we are investigating this issue. 

\paragraph{APIs and defining operations} 
%This analysis requires skimming and slimming operation. 
%The skimming and slimming operation involves the use of functions; \texttt{map}, \texttt{filter} and 
%\texttt{reduce}. 
Scala is a natural choice to implement this operation; Spark is written in Scala 
and it should out-perform Python on large data sets.  
 There are functions available to
  perform transformations, aggregations, global reduction in a distributed environment, which can be
  readily used. Such a set-up provides ease-of-programming, however, it does mean that the user has to
  rely on system optimizations provided by Spark's implementation to
  improve any performance. 
Similarly, it is hard to understand a DAG created by Spark for SQL queries on 
DataFrames, and users have minimal control over optimizing the 
query structure, data partitions, etc. In MPI, uses have to provide all the functionality keeping 
parallel processing in mind. 


\paragraph{Performance: } 

MPI implementation always performs better as compared with the Spark 
implementation for the test we ran using global summation across a column; 
both implementations are reading the same data, and doing the exact same calculation. 
We also observed that once data is loaded into the Spark DataFrame, the sum operation takes 
less than 4 seconds to complete. 
 
\paragraph{Scaling: }
The Spark implementation provided good
  scaling without requiring any tuning to the implementation and
  developer expertise in parallel algorithms. 
  When the number of partitions is less than number of cores allocated, 
we are wasting resources and observe degraded performance. 
However, it has been challenging to control number of tasks in some cases. 
We anticipate good scaling behavior when we use bigger data sets ($\approx$ 200TB). 
   

 \paragraph{Orchestration: } 
In Spark, data distribution and task assignment
  is abstracted from the user. There are functions available to
  perform global reduction in a distributed environment, which can be
  readily used.  Such a set-up provides ease-of-programming in a
  distributed environment. It does mean, however, that the user has to
  rely on system optimizations provided by Spark's implementation to
  improve any performance.
  We defined partition size, and number 
of tasks for our use case but for many use cases we can use the 
default partition size, which is equal to file system block size. 
But figuring out the best partition size for a job is challenging. 
Allocating tasks and data partition to the worker nodes is abstracted 
from the user. Users don't have to change the number of tasks to match 
with the number of cores available. 

  \paragraph{Application tuning: } All the transformations in Spark
  are lazy, with delayed calculation of results. The transformations
  applied to the base dataset are remembered by the system
  and only computed when an action is carried out on the dataset.
  This design enables Spark to run more efficiently. 
  Due to this lazy evaluation, it is hard to isolate slow-performing tasks and report
  timing for different stages. 
  Without an expert help 
it is challenging to identify what configuration and options to use for 
an application. 
For example, memory per node, number of cores per node,
debug level, etc.  
  In MPI, since developer has more control over data and task assignment, it is 
  straightforward to measure performance of each compute and I/O operation.  

\paragraph{Debugging: } 
Distributed applications are challenging to debug. 
In Spark applications, 
most of the times error messages are misleading and need more work
just to understand the error message before working on
a solution. Use of graphical interface and history server can
be useful but when using DataFrame API, understanding
different stages and partitions is not straightforward. 
In MPI, 



\paragraph{Multiple language support: }
The availability of Python and R interface for end user analyses is also a key feature desired by the HEP community.


\paragraph{Documentation} 
Spark could be used more effectively with comprehensive documentation with 
reasonable examples. 


\section{Conclusion}
Spark is relatively new and emerging technology, and its use, especially 
in the HEP community, is in exploratory stages. The learning curve involved 
with its use, especially using Scala, cannot be ignored. However, the availability 
of APIs in R and Python improves beginner's experience. 
Other advantages include task distribution and user controlled 
data partitioning. We have seen good scaling behavior of Spark applications 
with increase in dataset size and the number of nodes with no extra work. 
Encoding analysis workflow using Scala best practices and the optimal 
use of DataFrame features is challenging. The documentation 
and error reporting should be improved. However, the ease of use, 
reasonable performance and good scalability makes Spark a viable candidate for our future work.

Spark provides implicit parallelization of the
physicists' data processing algorithms using these functions, thus providing the possibility
of good scaling to large numbers of cores without requiring the
physicists to master complex parallel programming techniques. This
provides important ease of use for ``casual'' programmers, for whom
the goal is rapid turnaround time for analysis, who are usually not
interested in developing specialized parallel programming
skills. 

\textbf{Participation}: If accepted, Saba Sehrish will attend the conference. 

\textbf{Bio:} 
Saba Sehrish is a Computer Science Researcher 
at Fermi National Accelerator Laboratory. She did her post doc at Northwestern University
and PhD at University of Central Florida. 
Her research interests include HPC and Big data computing for HEP analysis.  
She is currently working on Spark for HEP and also a team member of the LArSoft 
project at Fermilab. 

\textbf{Collaborators: } 
Matteo Cremonesi, Oliver Gutsche, Jim Kowalkowski, 
Cristina Mantilla, Marc Paterno, Jim Pivarski, Alexey Svyatkovskiy

\textbf{Acknowledgement: }
We would like to thank Lisa Gerhardt for guidance in using Spark optimally at NERSC. 
This research supported through the  Contract No. DE-AC02-07CH11359 with the United States Department of Energy, 2016 ASCR Leadership Computing Challenge award titled ``An End-Station for Intensity and Energy Frontier Experiments and Calculations". This research used resources of the National Energy Research Scientific Computing Center, a DOE Office of Science User Facility supported by the Office of Science of the U.S. Department of Energy under Contract No. DE-AC02-05CH11231.
\scriptsize

%Bibliography and References Cited
\bibliography{earlycareerbib,ssio}

\end{document}
