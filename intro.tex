\section{Introduction}
\label{sec:intro}
\squeezeup

%Experimental High Energy Physics (HEP) deals with the understanding of fundamental particles and the interactions between them.
%Experimental HEP is a compute- and data-intensive statistical science; a large number of interactions 
%must be analyzed to discover new particles or to measure the properties of known particles. 
%For example, data from over $300$ trillion ($3\times10^{14}$) proton-proton collisions at the Large Hadron Collider (LHC) were analyzed for the Higgs boson discovery.
 %approximately $1$ in $10^{13}$ LHC collisions yielded a distinguishable Higgs boson~\cite{higgsboson1}. 
%The size of the data sample (25 petabytes per year) required the use of a worldwide computing grid, 
%comprised of 170 computing facilities across 42 countries~\cite{lhcgrid}. 
%Future HEP experiments will bring in even more data, and processing and 
%analyzing will be more challenging. For example, while the LHC generates up to a billion collisions per second; the High Luminosity-LHC~\cite{hllhc} will generate 5 times this rate. These larger data samples will be needed to obtain a 
%deeper understanding of the Higgs boson and its implications for the fundamental laws of nature. 
 
Experimental High Energy Physics (HEP) is a compute- and data-intensive statistical science; 
a large number of interactions ($\approx$petabytes of data)
must be analyzed to discover new particles or to measure the properties of known particles. 
For example, data from over $300$ trillion ($3\times10^{14}$) proton-proton collisions at the 
Large Hadron Collider (LHC) were analyzed for the Higgs boson discovery. 
While the LHC generates up to a billion collisions per second; the High Luminosity-LHC~\cite{hllhc} will generate 5 times this rate.
With increase in volume of data, the current tools and resources will be insufficient to 
provide low-latency and interactive analysis. 

HEP Analysis is an iterative process where low-latency and interactivity is the key to gain 
understanding of the fundamental particles and the interactions between them. 
The analyses are IO bound and consist of several intermediate steps requiring to store 
smaller reduced data sets on disk, which can be accessed interactively for end-user analysis. 
These intermediate steps consist of applying selection algorithms, calculations of statistical summaries, 
and exploratory plotting of the relevant summaries. 
These steps can take from days to weeks; 
storing intermediate results in files and associated data handling software infrastructure 
adds to the processing time. 

%CMS example
In the CMS Dark Matter search for new types of elementary particles,
the data volume to be analyzed is about ($\approx$ 200) terabytes for the 2015 dataset. 
 %and contains both data and Monte Carlo simulations. 
 This is expected to grow in the future significantly in LHC run 2. 
 The data is reduced to couple of gigabytes to facilitate end-user analysis.  
Similarly, in NOvA, an intensity frontier experiment, a particular analysis 
 performs selection on a billion events, and uses the passing 33 events with fewer 
 attributes to do parameter estimation and model comparison studies. 

We need to explore alternate approaches to in-memory data processing to 
improve time-to-physics and enable interactive analysis without significant overhead of bookkeeping and intermediate 
files. The new large-scale, state-of-the-art computing systems provide tremendous compute and memory resources, and look promising for changing the traditional ways of doing these data analyses. 
For example, Cori, supercomputer at NERSC provides aggregate memory of 203 terabytes for Haswell nodes and sustained application performance of 83 TFlop/s. 
However, the choice of programming abstractions to efficiently access and analyze data on these big compute machines remains a challenge. 
The traditional approach of using MPI burdens the programmer with fine grain details and control of the system to the level of data and task assignment to which memory technology to use. The new approach of using high level programming abstractions of Spark, an industry de-facto has been recently made available on these machines but is neither designed for the scientific applications and data nor to exploit the high performance computing platforms that are 
available to the scientific community.

In this talk, we present a thorough comparison of performance and usability aspects of 
both traditional and industry de-facto approach by implementing use cases from HEP.  
More specifically, we will focus on the challenges related to implementing selection and 
filtering criteria on billions of events using Apache Spark~\cite{spark,spark1} and MPI by running tests at NERSC~\cite{nersc-spark}. 
%Our work is unique because we have been using both MPI and Spark on HPC resources for HEP analyses. 